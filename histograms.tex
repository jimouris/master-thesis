\section{Privacy Preserving Histograms}\label{s:histograms}
Despite the challenges presented in sections \ref{s:challenges} and \ref{s:simple-algorithms}, the majority of algorithms can be transformed to their privacy preserving equivalent.
However, this is not a straightforward translation , as we examined in section \ref{s:simple-algorithms}, and in most cases it adds a complexity overhead to every algorithm that depends its control flow decisions in private data.

Histograms is a practical and notable example of algorithms that are widely used and the complexity of their privacy preserving version remains in computationally feasible levels, comparing to the textbook algorithm.
But first of all, what is a histogram?

As stated in \cite{ioannidis2003history}, histograms initially conceived as a visual aid to statistical approximations.
Webster’s defines a histogram as ``a bar graph of a frequency distribution in which the widths of the bars are proportional to the classes into which the variable has been divided and the heights of the bars are proportional to the class frequencies".
A histogram is generally a form of classifying and representing data in some categories of a specific range; the range is an individual ``base" element associated with each column.

More specifically, a histogram on some attributes $\{A, B, \dots, Z\}$ is constructed by partitioning the data distribution of those attributes into some ranges which are mutually disjoint subsets called buckets and approximating the frequencies and values in each bucket.
For the sake of simplicity let us suppose that the number of ranges ($\beta$) of all attributes are equal ($\beta \geq 1$).

In figure \ref{f:simple-hist} we present two histograms, the first one-dimensional over the attribute ``Patient Age" and the second is a two-dimensional over the attributes ``Patient Age" and ``Heart rate".
In both histograms $\beta = 4$, since the range of values for each attribute has been partitioned in four mutually disjoint subsets.
In the 1-dimensional histogram \ref{f:simple-hist}.a the $y-axis$ corresponds to the total number of occurrences of values that belong to each bucket.
In the 2-dimensional histogram \ref{f:simple-hist}.b since $y-axis$ corresponds to the ranges for the buckets for the second attribute, the occurrences are depicted with different colors.


\begin{figure}[h!]
    \centering
    \subfloat[1D Histogram]{\includegraphics[width=0.49\columnwidth]{figures/1d-simple-hist.png}}
    \subfloat[2D Histogram / Heatmap]{\includegraphics[width=0.49\columnwidth]{figures/2d-simple-hist.png}}

    \caption{(a) An one-dimensional histogram with $\beta = 4$
    (b) A two-dimensional histogram with $\beta = 4$.}
    \label{f:simple-hist}
\end{figure}


\fixme{in which cases is a histogram useful?}



\subsection{Agorithms for Privacy Preserving Histograms}\label{ss:histogram-algos}
Histograms are one of the simplest ways to visualise and easily understand data; rendering them very useful in many data analysis applications. 
As we already mentioned, histograms consist of bars that are mutually disjoint and their heights are proportional to the counts of values in the corresponding ranges.

\textbf{Challenges with Privacy-Preserving Histograms}: The textbook algorithm splits the dataset to buckets and consecutively for each item it increments the corresponing ``bucket-counter".
Implementing this algorithm with respect to the privacy of the dataset introduces the problem that for each value it is not trivial to find the bucket that this specific value should be placed.

As we mentioned in section \ref{s:two-types-of-data}, we have separated our algorithms in two major categories, for categorical and for numerical data.
In the following subsections we delve into details for the privacy preserving algorithms for both categories.



\fixme{Add somewhere text about vectorization. Probably in sharemind section.\\
Vectorization is used so that a possible value or range is compared to every value in the input data at once.
}


\subsubsection{Privacy Preserving Histograms for Categorical Values}\label{sss:histogram-categorical}
\textbf{One-Dimensional Histograms}: In algorithm \ref{a:1d-histogram-categorical} we present the privacy preserving algorithm of an one-dimensional (1D) histogram for categorical values.
In simple words, categorical data means that the values are discrete.
Hence, the second parameter in the algorithm (dubbed $P$) is the number of possible choices that exist in $array[N]$.
The input data is given in the form of a private array/vector.

The algorithm creates a boolean array of equal size as the input data, and then for each possible encrypted value of the dataset checks (line 3) and counts (line 4) its occurrences.
Method {\fontfamily{lmss} $\textsc{Sum}$} is the same as in algorithm \ref{a:sum}.
Finally, the counts are gathered into a vector and returned -- still encrypted -- to the user.

\import{./}{algorithms/1d_hist_categorical.tex}



\textbf{Multi-Dimensional Histograms}: In algorithm \ref{a:multidim-histogram-categorical} we present the privacy preserving algorithm of multi-dimensional histograms for categorical values.
As in algorithm \ref{a:1d-histogram-categorical}, here, the third parameter ($Ps$) is the number of possible choices that exist in $array$.
However, since here the input data is a private array with multiple dimensions $array[N][M]$, the possible values should express the possible values for each dimension.
Thus, it is an array of $A$ slots, where $A$ is the number of attributes.

Similarly to the 1D algorithm, the multi-dimensional one creates a boolean array of \fixme{explain the algorithm}

\import{./}{algorithms/multdim_hist_categorical.tex}


\textbf{Filtering the Histograms' Results}: In the preceding algorithms, the data 

\fixme{Add filters explanation. Maybe code?}



\subsubsection{Privacy Preserving Histograms for Numerical Values}\label{sss:histogram-numerical}

In algorithm \ref{a:1d-histogram-numerical} we present the privacy preserving algorithm of an one-dimensional (1D) histogram over numerical values.

\fixme{explain the algorithm}

\import{./}{algorithms/1d_hist_numerical.tex}




The algorithm that computes a private multi-dimensional histogram is similar to the one regarding one-dimensional histograms but also addresses the issue of having a histogram of arbitrarily many dimensions.
In case we had a known number of dimensions the simplest solution would be to use nested loops, as many as the histogram dimensions.
Since the number of dimensions is not known, we have to think of something else.
We represent the multi-dimensional histogram as a serialized version with an one-dimensional array (a vector) instead of using a multi-dimensional array (a matrix).
For example a 2-dimensional $3 \times 4$ histogram will be represented as a vector whose length is $ 12 $ ($= 3 \cdot 4$), and a $3 \times 4 \times 5$ 3-dimensional histogram wit a vector of length $ 60 $ ($= 3 \cdot 4 \cdot 5$)

When it comes to indexing if we wish to access the 2-dimensional $N \times M$ histogram represented as an one-dimensional at row $ i $ and column $ j $, instead of using $h[i][j]$ we use $h[i \cdot M + j]$.
Similarly, for the 3-dimensional $L \times N \times M$ histogram, $h[i][j][k]$ becomes $h[i \cdot N \cdot M + j \cdot M + k]$. So there needs to be a computation of a single index based on the multiple dimension indexes. In algorithm \ref{a:multidim-histogram-numerical}, the \texttt{positions} vector holds this index for each record of the provided array, as can bee seen in lines $ 12 $ and $ 13 $

\import{./}{algorithms/multdim_hist_numerical.tex}


Filters here are easily applied as in \ref{sss:histogram-categorical} ... \fixme{some exaplanation ..}


