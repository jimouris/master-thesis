\section{Privacy Preserving Histograms}\label{s:histograms}
Despite the challenges presented in section \ref{s:challenges}, the majority of algorithms can be transformed to their privacy preserving equivalent.
However, this is not a straightforward translation and in the most cases it adds a complexity overhead to every algorithm that depends its control flow decisions in private data.

Histograms is a practical and notable example of algorithms that are widely used and the complexity of their privacy preserving version remains in computationaly feasible levels, compairing to the textbook algorithm.
% A histogram is generally a form of classifing and representing data in some categories of a specific range.

\fixme{what is a histogram?}


\fixme{in which cases is a histogram useful?}




In algorithm \ref{a:1d-histogram-categorical} we present the privacy preserving algorithm of an one-dimensional (1D) histogram over categorical values.

\import{./}{algorithms/1d_hist_categorical.tex}

\fixme{explain the algorithm}



In algorithm \ref{a:1d-histogram-numerical} we present the privacy preserving algorithm of an one-dimensional (1D) histogram over numerical values.

\import{./}{algorithms/1d_hist_numerical.tex}

\fixme{explain the algorithm}



\import{./}{algorithms/multdim_hist_categorical.tex}

\fixme{explain the algorithm}



\import{./}{algorithms/multdim_hist_numerical.tex}

The algorithm computing a private multi-dimensional histogram is similar to the one regarding one-dimensional histograms but also addresses the issue of having a histogram of arbitrarily many dimensions.
In case we had a known number of dimensions the simplest solution would be to use nested loops, as many as the histogram dimensions.
Since the number of dimensions is not known, we have to think of something else.
We represent the multi-dimensional histogram as a serialized version with an one-dimensional array (a vector) instead of using a multi-dimensional array (a matrix).
For example a 2-dimensional $3 \times 4$ histogram will be represented as a vector whose length is $ 12 $ ($= 3 \cdot 4$), and a $3 \times 4 \times 5$ 3-dimensional histogram wit a vector of length $ 60 $ ($= 3 \cdot 4 \cdot 5$)

When it comes to indexing if we wish to access the 2-dimensional $N \times M$ histogram represented as an one-dimensional at row $ i $ and column $ j $, instead of using $h[i][j]$ we use $h[i \cdot M + j]$.
Similarly, for the 3-dimensional $L \times N \times M$ histogram, $h[i][j][k]$ becomes $h[i \cdot N \cdot M + j \cdot M + k]$. So there needs to be a computation of a single index based on the multiple dimension indexes. In algorithm \ref{a:multidim-histogram-numerical}, the \texttt{positions} vector holds this index for each record of the provided array, as can bee seen in lines $ 12 $ and $ 13 $



