\section{Privacy Preserving Histograms}\label{s:histograms}
Despite the challenges presented in section \ref{s:challenges}, the majority of algorithms can be transformed to their privacy preserving equivalent.
However, this is not a straightforward translation and in the most cases it adds a complexity overhead to every algorithm that depends its control flow decisions in private data.

Histograms is a practical and notable example of algorithms that are widely used and the complexity of their privacy preserving version remains in computationaly feasible levels, compairing to the textbook algorithm.
% A histogram is generally a form of classifing and representing data in some categories of a specific range.

\fixme{what is a histogram?}


\fixme{in which cases is a histogram useful?}




In algorithm \ref{a:1d-histogram-categorical} we present the privacy preserving algorithm of an one-dimensional (1D) histogram over categorical values.

\import{./}{algorithms/1d_hist_categorical.tex}

\fixme{explain the algorithm}



In algorithm \ref{a:1d-histogram-numerical} we present the privacy preserving algorithm of an one-dimensional (1D) histogram over numerical values.

\import{./}{algorithms/1d_hist_numerical.tex}

\fixme{explain the algorithm}


\import{./}{algorithms/multdim_hist_numerical.tex}
