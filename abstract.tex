\abstractEn{
The new era of big data demands high performance computing, since the amount of data published online is growing exponentially.
Cloud computing has emerged as a result, providing strong computational power for both individuals and companies.
Though cloud computing is the answer to many business models, there are many use\hyp cases where cloud fails to meet the demands of information privacy.
For instance, exposing financial and medical information to the cloud may violate the individuals' right to privacy.
People are not comfortable sharing their sensitive data, and more importantly, they do not trust any cloud provider with this information; data that are uploaded in the cloud can be exposed to attacks from both the cloud provider and third parties.


Nevertheless, there are many real world use cases that use information from different parties to jointly compute meaningful results, but due to the aforementioned limitations, some are avoided and others do not always respect data privacy.
The solution to this is a technique called Secure Multi\hyp Party Computation (SMPC or MPC), which leverages cryptographic primitives to carry out computations on confidential data, computing a function and learning nothing more than what the $N$ parties would have if a separate trusted party had collected their inputs, computed the same function for them, and then return the result to all parties.


Motivated by this wide range of applications, in this thesis we have focused on providing an end\hyp to\hyp end infrastructure for computing privacy\hyp preserving analytics.
More specifically, we have developed algorithms specifically tailored to encrypted architectures and in the SMPC scenario, such as secure aggregators and secure decision tree classifiers.
Moreover, we have focused on the coordination and communication between all involved parties; those who provide their data, those who perform the secure computation, and finally those that initiate new computations.
Our algorithms are not dependent to the application that our systems serves, however, in order to demonstrate it, in this thesis we use hospitals as data providers and we focus on medical research.
Our goal is to establish an end\hyp to\hyp end system for discovering useful information with respect to data privacy, and also to provide the building blocks for potentially more elaborate privacy\hyp preserving algorithms.
}


\abstractGr{
Η νέα εποχή των μεγάλων δεδομένων απαιτεί μεγάλη υπολογιστική ισχύ, αφού το πλήθος των δεδομένων που δημοσιεύονται στο διαδίκτυο μεγαλώνει εκθετικά.
Σαν αποτέλεσμα, προέκυψαν τα Νέφη Υπολογιστικών Συστημάτων, παρέχοντας μεγάλη υπολογιστική ισχύ, τόσο για ιδιώτες όσο και για επιχειρήσεις.
Παρόλο που τα υπολογιστικά νέφη είναι η απάντηση σε πολλά επιχειρηματικά μοντέλα, υπάρχουν πολλές περιπτώσεις χρήσης όπου τα υπολογιστικά νέφη αποτυγχάνουν να καλύψουν τις απαιτήσεις ιδιωτικότητας των πληρο\hyp φοριών.
Για παράδειγμα, εκθέτοντας οικονομικές και ιατρικές πληροφορίας στο νέφος μπορεί να παραβιάζει το δικαίωμα των ατόμων στην ιδιωτικότητα.
Οι άνθρωποι δεν νιώθουν άνετα με το να μοιράζονται τα ευαίσθητα δεδομένα τους, και πιο σημαντικά, δεν εμπιστεύονται κανέναν πάροχο υπολογιστικού νέφους με τις πληροφορίες αυτές.
Τα δεδομένα που μεταφορτώνονται στο νέφος μπορεί να εκτεθούν σε επιθέσεις τόσο από τον πάροχο όσο και από τρίτους.

Παρόλα αυτά, υπάρχουν πολλές πραγματικές περιπτώσεις χρήσης που χρησιμοποιούν πληροφορίες από διαφορετικές οντότητες προκειμένου να υπολογίσουν από κοινού ουσι\hyp αστικά αποτελέσματα, αλλά λόγω των προαναφερθέντων περιορισμών, κάποιες από αυτές αποφεύγονται και άλλες δεν σέβονται πάντα την ιδιωτικότητα των δεδομένων.
Η λύση σε αυτό είναι μία τεχνική που ονομάζεται Ασφαλής Υπολογισμός Πολλαπλών Συμμετεχόντων, η οποία αξιοποιεί θεμελιώδεις κρυπτογραφικές ιδιότητες προκειμένου να εκτελέσει υπολογισ\hyp μούς πάνω από εμπιστευτικά δεδομένα, υπολογίζοντας μία συνάρτηση και μαθαίνοντας τίποτε παραπάνω σε σχέση με το τι θα μάθαιναν $N$ συμμετέχοντες, εαν μία ξεχωριστή έμπιστη οντότητα είχε συλλέξει τις εισόδους τους, είχε εκτελέσει την ίδια συνάρτηση, και τέλος επέστρεφε το αποτέλεσμα σε όλους τους συμμετέχοντες.

Παίρνοντας κίνητρο από αυτό το ευρύ φάσμα εφαρμογών, στην εργασία αυτή επικεντρωθή\hyp καμε στο να παρέχουμε μία υποδομή από άκρη σε άκρη για υπολογισμό στατιστικών αναλύσεων με σεβασμό στην ιδιωτικότητα.
Πιο συγκεκριμένα, έχουμε υλοποιήσει αλγορίθ\hyp μους ειδικά σχεδιασμένους για κρυπτογραφημένες αρχιτεκτονικές, χρησιμοποιώντας το σενάριο του Ασφαλή Υπολογισμού Πολλαπλών Συμμετεχόντων, όπως ασφαλείς συγκεντρωτ\hyp ικούς αλγορίθμους και ασφαλείς κατηγοριοποιητές με δέντρα απόφασης.
Ακόμα, συγκεντ\hyp ρωθήκαμε στο συντονισμό και την επικοινωνία μεταξύ όλων των συμμετεχόντων. Αυτών που παρέχουν δεδομένα, αυτών που εκτελούν τον ασφαλή υπολογισμό και τέλος αυτών που ξεκινούν νέους υπολογισμούς.
Οι αλγόριθμοί μας δεν εξαρτώνται από την εφαρμογή που εξυπηρετεί το σύστημά μας, παρόλα αυτά, για λόγους παρουσίασης, στην εργασία αυτή χρησιμοποιούμε νοσοκομεία σας παρόχους δεδομένων και επικεντρωνόμαστε στην ιατρική έρευνα.
Ο Στόχος μας είναι να ιδρύσουμε ένα ολοκληρωμένο σύστημα με σκοπό την ανακάλυψη χρήσιμης πληροφορίας με σεβασμό στην ιδιωτικότητα, και επίσης να προσφέρουμε τα δομικά στοιχεία για τυχόν πιο πολύπλοκους αλγορίθμους με σεβασμό στην ιδιωτικότητα.
}
