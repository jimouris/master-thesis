\section{Symmetric-Key Encryption}\label{s:symmetric-key-encryption}
Symmetric-key algorithms are algorithms that use the same cryptographic keys for both encryption of plaintext and decryption of ciphertext.
The keys represent a shared secret between two or more parties that can be used to maintain a private information channel.
This requirement that both parties have access to the secret key is one of the main drawbacks of symmetric key encryption, in comparison to public-key encryption schemes that we expand in section \ref{s:public-key-encryption}.

More formally, given a plaintext p and a secret key $k$ a ciphertext $c$ is produced such that:
$c = enc(p, k), and p = dec(c, k) = dec(enc(p, k), k).$

This form of encryption has been widely adopted by numerous applications, and can be applied either for data at rest (e.g. an encrypted database), or data in transit (e.g. TLS/SSL protocol).
More specifically, the definition data at rest implies data that is stored physically in any digital form, while data in transit means data traversing the network.
There is also a third state of digital information, namely data in use, i.e. active data under constant change which is stored in a non-persistent digital state typically in computer random access memory (RAM), CPU caches, or CPU registers.
Thus, the data in use can be exposed to security risks, including side-channel attacks [...], as well as hardware Trojans [...], which can leak sensitive information.
