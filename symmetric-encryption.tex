\section{Symmetric-Key Encryption}\label{s:symmetric-key-encryption}
The simplest form of encryption, is the use of a symmetric key scheme. Such a scheme utilizes a secret key which is used to perform the mathematical operations of both encrypting and decrypting of a message.
The communicating parties must share the same key in order to communicate securely.
Symmetric-key algorithms are algorithms that use the same key for both encryption of the plaintext and decryption of the ciphertext.
The keys represent a shared secret between two or more parties that can be used to maintain a private information channel.
The requirement that both parties have access to the secret key is one of the main drawbacks of symmetric key encryption, in comparison to public-key encryption schemes that we expand in section \ref{s:public-key-encryption}.

More formally, given a plaintext $p$ and a secret key $k$, a ciphertext $c$ is produced such that:
\begin{equation}
  \label{eq:t}
  \begin{aligned}
    c = enc(p, k)\\
    p = dec(c, k) = dec(enc(p, k), k)
  \end{aligned}
  \end{equation}
Where $enc$ and $dec$ are the \textit{Encryption} and \textit{Decryption} algorithms respectively.
