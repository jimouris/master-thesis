\section{Asymmetric-Key Encryption (Public-Key Encryption)}\label{s:public-key-encryption}
The major problem that traditional symmetric-key algorithms face is that they require the parties to exchange keys, or more specifically to agree to a secret common private key, prior to communicating.
This needs a secure (physical or digital) communication channel accessible even for a short period of time, in order for the secret key to be exchanged. Such a channel could be a piece of paper delivered by hand or by a trusted courier, or a short in person communication.

An asymmetric or public-key cryptosystem is one where different keys are employed for the operations in the cryptosystem (e.g., encryption and decryption), and where one of the keys can be made public without compromising the secrecy of the other key \cite{Kaliski2011}. That way there is no need for a secure channel or prior agreement.

In the classic model such an encryption scheme involves a \textit{public key} which is used for encryption and a \textit{private key} which is used for decryption.
More formally, given a plaintext $P$, a public key $PubKey$, and a private key $PrivKey$ we can produce a ciphertext $C$ such that:
\begin{equation}
  \label{eq:public-key-encryption}
  \begin{aligned}
    C = Enc(P, PubKey)\\
    P = Dec(C, PrivKey)
  \end{aligned}
  \end{equation}

Where $Enc$ and $Dec$ are the \textit{Encryption} and \textit{Decryption} algorithms respectively.

If Alice wants to send a message to Bob, she needs Bob's public key which could be publicly available (e.g. on Bob's website).
Alice encrypts her message with Bob's public key and sends the resulted ciphertext to Bob. Bob then needs his private key which is only in his possession in order to decrypt the ciphertext Alice sent and retrieve the original message.
If Bob wishes to reply, he needs Alice's public key etc.
Generally, the public (encryption) key could be shared to any party that wants to communicate with the owner of that public key, while the private (decryption) key should be kept secret and used by the recipient of the ciphertext to recover the original plaintext.

The ability for two users to establish a shared secret over an insecure communication channel, despite having no prior communication or information exchange was first proposed in 1976 by Whitfield Diffie and Martin Hellman.
That method is named Diffie-Hellman (DH) key exchange \cite{diffie1976new} after its authors.

\fixme{
More formally, a public key cryptosystem is made up of three algorithms:
A Key Generation algorithm that is given a security parameter η and returns the public and private keys.
An Encryption algorithm that is given a message, some randomness and the public key and returns the encrypted message.
As we shall see in 2.1.3 any deterministic public key cryptosystem cannot be secure.
To this end, randomness is essential to a public key encryption function.
A Decryption algorithm that given the ciphertext and the private key recovers the plaintext.
}

\fixme{
Public key cryptography accomplishes two functions: authentication, where the public key verifies that a holder of the paired private key sent the message, and encryption, where only the paired private key holder can decrypt the message encrypted with the public key.
}

Some of the most widely used public key cryptosystems are the RSA (Rivest -- Shamir -- Adleman) \cite{rivest1978method}, the El Gamal \cite{elgamal1985public} and the Paillier \cite{paillier1999public} Cryptosystems.
\fixme{The importance of public-key Encryption schemes can also be understood by the plethora of schemes and algorithms that have been inspired from it, like the Digital Signature Algorithm and OpenPGP that serve varied purposes.}
