\section{Secret Sharing}\label{s:secret-sharing}
Secret sharing is a method for distributing a secret between $N$ different parties, where a secret message $m$ is divided into parts, giving each participant its own unique part $(s_1, s_2, \dots, s_N)$.
Combining some of the parts or all of them are needed in order to reconstruct the original secret.
The most common type of secret sharing is a scheme with one dealer and $N$ different players.
Initially, the dealer splits the secret to shares and gives each player a split.
Only with all -- or most of the shares $k \geq T$  ($T$ is a threshold) -- someone is able to reconstruct the original secret.
The dealer accomplishes this by giving each player a share in such a way that any group of $T$ or more players can together reconstruct the secret but no group of fewer than $T$ players can.
Such a system is called a $(T, N)$ - threshold scheme.

A naive secret sharing scheme between a dealer and two parties is to apply bitwise \texttt{XOR} to the secret and a random string of the same length.
For the sake of simplicity, let us assume that the secret message m is a string of 5 bits (either 0 or 1), \textit{e.g.} 01101.
Then, the dealer generates a random string of 5 bits $s_1$ (\textit{e.g.} $10011$) and performs the \texttt{XOR} operation between $s_1$ and $m$; a string $s_2 = 11110$ arises.
Finally, the dealer gives the $s_1$ and $s_2$ to the two parties.
Neither one of them is able to retrieve the secret message $m$ without the other half share, and also nor is able to retrieve any information about the original message.

There exist more complex and elaborated systems for more than two parties, such as Shamir’s \cite{shamir1979share}, Blakley’s \cite{blakley1994linear}, or Additive \cite{kim2003designs} secret sharing schemes.


\subsection{Shamir Secret Sharing}\label{ss:shamir-secret-sharing}
Secret sharing was proposed in \cite{shamir1979share} by Shamir, and it was the first method to enable distribution of a secret to $N$ parties.
A secret message $m$ is divided into $N$ parts, giving each participant its own unique part, where any $T$ subset of $N$ can recover the secret, but no $T-1$ element subset can.

Adi Shamir's threshold scheme is based on the idea of polynomial interpolation -- the interpolation of a given dataset by the polynomial of lowest possible degree that passes through the points of the dataset --, using Lagrange coefficients.
Shamir's secret sharing scheme constructs a polynomial $P$ of degree $N-1$, where $N$ is the number of players, as follows.

\begin{itemize}
  \item Dealer chooses a random polynomial of degree $N − 1$ so that $P(0) = m$, where $m$ is the secret message, and also the number $T$ of sufficient subsets that can reconstruct the secret.

  \textit{For example\footnote{This example is based on a similar example in \url{https://en.wikipedia.org/wiki/Shamir's_Secret_Sharing} Wikipedia link.}}, let us suppose our secret message $m = 1234$ and $N = 6$ parties, but any subset of $T = 3$ is sufficient to reconstruct the secret.
  Then our polynomial should have degree 2, $f(x) = ax^2 + bx + m$.
  The dealer chooses $N - 1$ random numbers, \textit{i.e.} $a = 94$ and $b = 166$, thus our polynomial is $f(x) = 94x^2 + 166x + 1234$.


  \item Dealer distributes $N$ pairs $(x_i , P(x_i)),$ $x_i \neq 0$

  First the dealer creates the $N$ pairs.
  \textit{For example}:\\
  $D_0 = (1, 1494)$, $D_1 = (2, 1942)$, $D_2 = (3, 2578)$, $D_3 = (4, 3402)$, $D_4 = (5, 4414)$, $D_5 = (6, 5614)$ and consecutively he distributes them to the $N$ parties.


  \item $N$ players can reconstruct the polynomial $P$ with their pairs, however $N - 1$ cannot

  In order to reconstruct the secret any $T$ points will be enough, \textit{for instance}, $D_1, D_3$ and $D_4$.
  Computing the Lagrange polynomials:\\
  $l_0 = \cfrac{x - x_1}{x_0 - x_1} \times \cfrac{x - x_2}{x_0 - x_2} = \cfrac{x - 4}{2 - 4} \times \cfrac{x - 5}{2 - 5} = \cfrac{1}{6}x^2 - \cfrac{3}{2}x + \cfrac{10}{3}$

  $l_1 = \cfrac{x - x_0}{x_1 - x_0} \times \cfrac{x - x_2}{x_1 - x_2} = \cfrac{x - 2}{4 - 2} \times \cfrac{x - 5}{4 - 5} = \cfrac{1}{2}x^2 + \cfrac{7}{2}x - 5$

  $l_2 = \cfrac{x - x_0}{x_2 - x_0} \times \cfrac{x - x_1}{x_2 - x_1} = \cfrac{x - 2}{5 - 2} \times \cfrac{x - 4}{5 - 4} = \cfrac{1}{3}x^2 - 2x + \cfrac{8}{3}$

  Therefore:

  $f(x) = \sum_{i=0}^{2} y_i \times l_i(x) = 94x^2 + 166x + 1234$

  Each party can compute $f(0)$ in order to obtain the secret, in this case $f(0) = 1234$.
\end{itemize}

It is evident, that in Shamir's scheme two points are sufficient to define a line $f(x) = ax + b$, three points are sufficient to define a parabola $f(x) = ax^2 + bx + c$, four points to define a cubic curve $f(x) = ax^3 + bx^2 + cx + d$ and so forth.
That is, it takes $T$ points to define a polynomial of degree $T-1$, where $T - 1$ will be the sufficient number of parties that can reconstructed the secret message.




\subsection{Additive Secret Sharing}\label{ss:additive-secret-sharing}
\fixme{
Secret sharing schemes can tolerate an adversary controlling up to t parties out of n total parties, where t varies based on the scheme, the adversary can be passive or active, and different assumptions are made on the power of the adversary.
The Shamir secret sharing scheme is secure against a passive adversary when $t < \cfrac{n}{2}$ and an active adversary when $t < \cfrac{n}{3}$ while achieving information-theoretic security, meaning that even if the adversary has unbounded computational power, they cannot learn any information about the secret underlying a share.
The BGW protocol, which defines how to compute addition and multiplication on secret shares, is often used to compute functions with Shamir secret shares.
Additive secret sharing schemes can tolerate the adversary controlling all but one party, that is $t < n$, while maintaining security against a passive and active adversary with unbounded computational power. Some protocols require a setup phase, which may only be secure against a computationally bounded adversary.
}

\fixme{
A number of systems have implemented various forms of MPC with secret sharing schemes. The most popular is SPDZ[15], which implements MPC with additive secret shares and is secure against active adversaries.
}



\subsection{Secret Sharing Homomorphism}\label{ss:secret-sharing-homomorphism}
\cite{benaloh1986secret}


