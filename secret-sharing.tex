\section{Secret Sharing}
Secret sharing is a method  for distributing a secret between N different parties, where a secret message m is divided into parts, giving each participant its own unique part (s1, s2, …, sN).
Combining some of the parts or all of them are needed in order to reconstruct the original secret. The most common type of secret sharing is a scheme with one dealer and N different players.
Initially, the dealer splits the secret to shares and gives each player a split. Only with all – or most of the shares k >= T  (T is a threshold) – someone is able to reconstruct the original secret.
The dealer accomplishes this by giving each player a share in such a way that any group of T or more players can together reconstruct the secret but no group of fewer than T players can. Such a system is called a (T, N) – threshold scheme.

A naive secret sharing scheme between a dealer and two parties is to apply bitwise XOR to the secret and a random string of the same length.
For the sake of simplicity, let us assume that the secret message m is a string of 5 bits (either 0 or 1), e.g. 01101.
Then, the dealer generates a random string of 5 bits s1 (e.g. 10011) and performs the XOR operation between s1 and m; a string s2 = 11110 arises.
Finally, the dealer gives the s1 and s2 to the two parties.
Neither one of them is able to retrieve the secret message m without the other half share, and also nor is able to retrieve any information about the original message.

There exist more complex and elaborated systems for more than two parties, such as Shamir’s [x], Blakley’s [y], or Additive [z] secret sharing schemes, but we will not delve into details about them.
