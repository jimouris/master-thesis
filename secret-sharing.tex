\section{Secret Sharing}\label{s:secret-sharing}
Secret sharing is a method for distributing a secret between $N$ different parties, where a secret message $m$ is divided into parts, giving each participant its own unique part $(s_1, s_2, \dots, s_N)$.
Combining some of the parts or all of them are needed in order to reconstruct the original secret.
The most common type of secret sharing is a scheme with one dealer and $N$ different players.
Initially, the dealer splits the secret to shares and gives each player a split.
Only with all -- or most of the shares $k \geq T$  ($T$ is a threshold) -- someone is able to reconstruct the original secret.
The dealer accomplishes this by giving each player a share in such a way that any group of $T$ or more players can together reconstruct the secret but no group of fewer than $T$ players can.
Such a system is called a $(T, N)$ - threshold scheme.

A naive secret sharing scheme between a dealer and two parties is to apply bitwise \texttt{XOR} to the secret and a random string of the same length.
For the sake of simplicity, let us assume that the secret message m is a string of 5 bits (either 0 or 1), \textit{e.g.} 01101.
Then, the dealer generates a random string of 5 bits $s_1$ (\textit{e.g.} $10011$) and performs the \texttt{XOR} operation between $s_1$ and $m$; a string $s_2 = 11110$ arises.
Finally, the dealer gives the $s_1$ and $s_2$ to the two parties.
Neither one of them is able to retrieve the secret message $m$ without the other half share, and also nor is able to retrieve any information about the original message.

There exist more complex and elaborated systems for more than two parties, such as Shamir’s \cite{shamir1979share}, Blakley’s \cite{blakley1994linear}, or Additive \cite{kim2003designs} secret sharing schemes.


\subsection{Shamir Secret Sharing}\label{ss:shamir-secret-sharing}
Secret sharing was proposed in \cite{shamir1979share}, and was the first method to enable distribution of a secret to $N$ parties.
The objective of the scheme is to share a secret message $m$ between $N$ different players so that any $T$ subset can recover it, but no $T-1$ element subset can.

Adi Shamir's threshold scheme is based on the idea of polynomial interpolation, using Lagrange coefficients.
For instance, in Shamir's scheme 2 points are sufficient to define a line, 3 points are sufficient to define a parabola, 4 points to define a cubic curve and so forth.
That is, it takes $k$ points to define a polynomial of degree $k-1$.

\fixme{a lot more to explain..}


\subsection{Additive Secret Sharing}\label{ss:additive-secret-sharing}


\subsection{Secret Sharing Homomorphism}\label{ss:additive-secret-sharing}
\cite{benaloh1986secret}


