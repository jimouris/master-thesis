\chapter{Implementation Details}\label{c:implementation}

Our application consists of four concrete parts, the \textit{coordinator}, the data providers, the SMPC cluster and the user interface.

\section{Coordinator}\label{s:impl-coordinator}
The \textit{coordinator} handles all private computation requests, communicates with the data providers -- requesting them to securely import their data to the computing cluster --, and finally returns to the researcher the analytics results.

The coordinator server has been implemented using \texttt{Node.js Express Server} and provides the following RESTful API.


\subsection{RESTful API}\label{ss:coord-restful-api}

\begin{table}[H]
\centering
\caption{Coordinator's RESTful API}
\label{t:coordinator-api}
\begin{tabular}{@{}ll@{}}
\toprule
URI                                   & Method        \\ \midrule
\texttt{/smpc/histogram/numerical}    & \texttt{POST} \\
\texttt{/smpc/histogram/categorical}  & \texttt{POST} \\
\texttt{/smpc/decisionTree}           & \texttt{POST} \\
\texttt{/smpc/queue?=request=\{uid\}} & \texttt{GET}  \\ \bottomrule
\end{tabular}
\end{table}

The RESTful API of the \textit{coordinator} is summarized in table \ref{t:coordinator-api}.
Below we explain in more detail each individual POST and GET request.

\begin{itemize}
\item \textbf{\textit{/smpc/histogram/numerical}}
This POST request initiates a secure computation for numerical/continuous values dataset.
Through the request's body, one can specify the desired attribute names of the dataset.
For instance, for the CVI dataset \ref{s:datasets} a user specifies that he/she wants to compute a histogram with $\beta = 4$ for term ``Patient Age''.
If the specified attributes are two or more, then the resulting counts will correspond to tuples, triples, etc.

{
\begin{minted}[xleftmargin=21pt, framesep=3mm, frame=single, linenos, tabsize=2, breaklines, breaksymbolleft=, fontsize=\footnotesize]{js}
{
  "attributes": [
    { "name": "Patient Age", "cells": "5" },
    { "name": "Height (cm)", "cells": "3" },
  ],
  "datasources": [
    "HospitalA",
    "HospitalB"
  ]
}
\end{minted}
\captionof{lstlisting}{Example /smpc/histogram/numerical POST request body}
\label{sc:histogram-numerical-post}
}



\item \textbf{\textit{/smpc/histogram/categorical}}
This POST request initiates a secure computation for categorical values dataset (MeSH).
Through the request's body, one can specify the desired MeSH terms.
The values the count of which will be computed are the children of the specified MeSH term from the MeSH ontology.
For example, if a user specifies that he/she wants the counts for the MeSH term ``Age Groups" [M01.060], he/she will get four counts back corresponding to the four children of ``Age Groups", namely ``Adolescent" [M01.060.057], ``Adult" [M01.060.116], ``Child" [M01.060.406] and ``Infant" [M01.060.703].
If the specified MeSH terms are two, then the resulting counts will correspond to tuples of MeSH labels.
If the specified MeSH terms are three, the result will be triples, etc.

{
\begin{minted}[xleftmargin=21pt, framesep=3mm, frame=single, linenos, tabsize=2, breaklines, breaksymbolleft=, fontsize=\footnotesize]{js}
{
  "attributes": [
      "M01.060" // This is a comma separated list of MeSH terms
  ],
  "datasources": [
    "HospitalA",
    "HospitalB"
  ]
}
\end{minted}
\captionof{lstlisting}{Example /smpc/histogram/categorical POST request body}
\label{sc:histogram-categorical-post}
}



\item \textbf{\textit{/smpc/decisionTree}}
Finally, this POST request is for computing a decision tree from either dataset.
The implemented classifiers are ID3 and C4.5, and one should specify which to use in the request body along with the desired attributes.
Moreover, he/she should specify the class-attribute, according to which the decision tree will occur.

{
\begin{minted}[xleftmargin=21pt, framesep=3mm, frame=single, linenos, tabsize=2, breaklines, breaksymbolleft=, fontsize=\footnotesize]{js}
{
  "attributes": [
    { "name": "Patient Age", "cells": "5" },
    { "name": "Heart rate", "cells": "3" },
    { "name": "Height (cm)", "cells": "6" }
  ],
  "classifier": "C45",
  "class_attribute": { "name": "RVESV (ml)", "cells": "3" },
  "datasources": [
    "HospitalA",
    "HospitalB"
  ]
}
\end{minted}
\captionof{lstlisting}{Example /smpc/decisionTree POST request body for numerical values}
\label{sc:decisionTree-post-numerical}
}

{
\begin{minted}[xleftmargin=21pt, framesep=3mm, frame=single, linenos, tabsize=2, breaklines, breaksymbolleft=, fontsize=\footnotesize]{js}
{
  "attributes": [
    "C14",
    "C12"
  ],
  "classifier": "ID3",
  "class_attribute": "M01.060",
  "datasources": [
    "HospitalA",
    "HospitalB"
  ]
}
\end{minted}
\captionof{lstlisting}{Example /smpc/decisionTree POST request body for categorical values}
\label{sc:decisionTree-post-categorical}
}




\item \textbf{\textit{/smpc/queue?=request=\{uid\}}}
This GET request is for checking the status and/or result of a specified computation request (the aforementioned POST requests).
The status of an ongoing computation request can be accessed through the /smpc/queue GET request by specifying its id.
The only parameter this GET request accepts is the id of the desired computation request, as shown below.

{
\begin{minted}[xleftmargin=21pt, framesep=3mm, frame=single, linenos, tabsize=2, breaklines, breaksymbolleft=, fontsize=\footnotesize]{js}
{
  "location": "/smpc/queue?request=944222c9-524b-40eb-b19b-486a946c220d"
}
\end{minted}
\captionof{lstlisting}{Example /smpc/queue GET request}
\label{sc:get}
}

The response is a JSON object containing the specified computation's status, and possibly its current step or result which is a JSON object too.
The server's response has the following structure.
\begin{description}[labelwidth=4em, leftmargin=\dimexpr\labelwidth+\labelsep\relax]
    \item [\texttt{status}:] A string indicating the computation's status. One of [running, succeeded, failed, notstarted]
    \item [\texttt{step}:] A string indicating the current step of the computation. This is present in case that the computation is in the running state.
    \item [\texttt{result}:] A JSON object with the computation's result in case its status is succeeded. The JSON object contains a single key namely data which contains tuples of (label, value) of the computation result.
    \begin{description}[labelwidth=4em, leftmargin=\dimexpr\labelwidth+\labelsep\relax]
        \item [\texttt{label}:] A string, the value name corresponding to that count. Can be a tuple, triple etc. depending on the number of queried Mesh terms.
        \item [\texttt{value}:] An integer, the actual count for that value.
    \end{description}
\end{description}


{
\begin{minted}[xleftmargin=21pt, framesep=3mm, frame=single, linenos, tabsize=2, breaklines, breaksymbolleft=, fontsize=\footnotesize]{js}
{
  "status": "succeeded",
  "result": {
      "data": [
          {
              "value": 0,
              "label": "Infant"
          },
          {
              "value": 63,
              "label": "Adolescent"
          },
          {
              "value": 9936,
              "label": "Adult"
          },
          {
              "value": 0,
              "label": "Child"
          }
      ]
  }
}
\end{minted}
\captionof{lstlisting}{Example /smpc/queue GET response body}
\label{sc:get-response}
}

\end{itemize}



\subsection{Sequence of Actions}\label{ss:coordinator-sequence}

\fixme{Explain the procedure in more detail}

\begin{enumerate}
\item A request arises in the coordinator
\item ...
\item main generator
\item compile
\item run
\item ...
\item plot or json
\item ...
\item send back the results
\end{enumerate}


\subsection{Coordinator Logging}\label{ss:coordinator-logging}



%%%%%%%%%%%%%%%%%%%%%%%%%%%%%%%%%%%%%%%%%%%%%%%%%%%%%%%%%%%%%%%%%%%%%%%%%%%%%%%%
%%%%%%%%%%%%%%%%%%%%%%%%%%%%%%%%%%%%%%%%%%%%%%%%%%%%%%%%%%%%%%%%%%%%%%%%%%%%%%%%

\section{Data Providers}\label{s:impl-data-providers}
When a private computation request arises, the \textit{coordinator} communicates with the data providers and requests them to securely import their data to the computing cluster.
Finally the \textit{coordinator} returns the results of the computation to the researcher.

Data providers listen to import requests from the \textit{coordinator} server using a RESTful API, implemented using \texttt{Node.js Express Server}.

\fixme{Specify that a user cannot request data-import, only the coordinator!}


\subsection{RESTful API}\label{ss:data-providers-restful-api}

\begin{table}[H]
\centering
\caption{Data Providers' RESTful API}
\label{t:data-providers-api}
\begin{tabular}{@{}ll@{}}
\toprule
URI                                & Method        \\ \midrule
\texttt{/smpc/import/numerical}    & \texttt{POST} \\
\texttt{/smpc/import/categorical}  & \texttt{POST} \\ \bottomrule
\end{tabular}
\end{table}


\begin{itemize}
\item \textbf{\textit{/smpc/import/numerical}}
This POST request initiates a secure import for numerical/continuous values dataset.
Through the request's body, one can specify the desired attribute names of the dataset to import.
For instance, an example for the CVI dataset \ref{s:datasets} is shown in Code \ref{sc:smpc-import-post}, in which a user specifies that he/she wants to compute private analytics for terms ``Patient Age'' and ``Heart rate''.

\item \textbf{\textit{/smpc/import/categorical}}
This POST request initiates a secure import for categorical values dataset (MeSH).
Through the request's body, one can specify the desired attribute names of the dataset to import.
The request for importing MeSH dataset is similar to the the numerical one, in Code \ref{sc:smpc-import-post}, however in the attributes should be given MeSH terms.

\end{itemize}

{
\begin{minted}[xleftmargin=21pt, framesep=3mm, frame=single, linenos, tabsize=2, breaklines, breaksymbolleft=, fontsize=\footnotesize]{js}
{
  "attributes": [ // Similarly, this works for MeSH terms
    "Patient Age",
    "Heart rate"
  ], // (i.e. "attributes": [ "C12", "C14", ... ])
  "datasource": "HospitalA_944222c9-524b-40eb-b19b-486a946c220d" // "datasource": Data-Provider-Name + "_" + {uid}
}
\end{minted}
\captionof{lstlisting}{Example /smpc/import/\{numerical, categorical\} POST request body}
\label{sc:smpc-import-post}
}


\subsection{Docker}\label{ss:data-providers-docker}


\subsection{Data Importer}\label{ss:data-providers-importer}


\subsubsection{Data Preprocessor}\label{sss:data-providers-preprocessor}



%%%%%%%%%%%%%%%%%%%%%%%%%%%%%%%%%%%%%%%%%%%%%%%%%%%%%%%%%%%%%%%%%%%%%%%%%%%%%%%%
%%%%%%%%%%%%%%%%%%%%%%%%%%%%%%%%%%%%%%%%%%%%%%%%%%%%%%%%%%%%%%%%%%%%%%%%%%%%%%%%

\section{SMPC Cluster}\label{s:impl-smpc-cluster}

\subsection{SMPC Servers}\label{ss:smpc-cluster-servers}




%%%%%%%%%%%%%%%%%%%%%%%%%%%%%%%%%%%%%%%%%%%%%%%%%%%%%%%%%%%%%%%%%%%%%%%%%%%%%%%%
%%%%%%%%%%%%%%%%%%%%%%%%%%%%%%%%%%%%%%%%%%%%%%%%%%%%%%%%%%%%%%%%%%%%%%%%%%%%%%%%

\section{User Interface}\label{s:impl-ui}

\fixme{Add screenshots}



\section{Communication}\label{s:impl-communication}

\fixme{Mention https}
\fixme{Mention promises}

