\chapter{Data Providers' RESTful API}\label{c:data-providers-api}

\section[/smpc/import/numerical POST request]{\protect\icon{figures/post4.pdf}}\label{s:post4}
This POST request initiates a secure import for numerical/continuous values dataset.
\begin{description}[labelwidth=5em, leftmargin=\dimexpr\labelwidth+\labelsep\relax]
\item[Request:] Through the request's body, one can specify the desired attribute names (\textit{i.e.} columns of the dataset) to be imported.
For instance, an example for the CVI dataset (see section \ref{c:datasets}) is shown in Code \ref{sc:smpc-import-cvi}, in which a user specifies that he/she wants to compute private analytics for attributes ``Patient Age'' and ``Heart rate''.

The request's body is a JSON object of the following form.

\begin{description}[labelwidth=6em, leftmargin=\dimexpr\labelwidth+\labelsep\relax]

    \item[\texttt{attributes}:] {\color{red}\textit{required}} A list of attributes (\textit{strings}) on which this histogram will be built.

    \item[\texttt{datasource}:] {\color{red}\textit{required}} The name for the table of the SPMC cluster's encrypted database in which the data  will be imported.

\end{description}

\ \\
\begin{minipage}{\linewidth}
  Example (importing of some attribute from the CVI dataset):\\
{
\begin{minted}[framesep=3mm, frame=single, tabsize=2, breaklines, breaksymbolleft=, fontsize=\footnotesize]{json}
{
    "attributes": [
        "Patient Age",
        "Heart rate"
    ],
    "datasource": "HospitalA"
}
\end{minted}
\captionof{lstlisting}{Example /smpc/import/numerical POST request body}
\label{sc:smpc-import-cvi}
}
\end{minipage}


\item[Response:] The server responds with \texttt{HTTP/1.1 200 OK} upon successful importing or with \texttt{HTTP/1.1 400 Bad Request} upon failure.
\end{description}




\section[/smpc/import/categorical POST request]{\protect\icon{figures/post5.pdf}}\label{s:post5}
This POST request initiates a secure import from datasets with categorical values dataset (MeSH).
\begin{description}[labelwidth=5em, leftmargin=\dimexpr\labelwidth+\labelsep\relax]
\item[Request:] Through the request's body, one can specify the desired attribute names of the dataset to import.
The request for importing attributes from th MeSH dataset is similar to the the numerical one, in Code \ref{sc:smpc-import-cvi}, however in attributes should be specified by their MeSH codes, due to name ambiguity.

The request's body is a JSON object of the following form.

\begin{description}[labelwidth=6em, leftmargin=\dimexpr\labelwidth+\labelsep\relax]

    \item[\texttt{attributes}:] {\color{red}\textit{required}} A list of attributes (MeSH codes) on which this histogram will be built.

    \item[\texttt{datasource}:] {\color{red}\textit{required}} The name for the table of the SPMC cluster's encrypted database in which the data  will be imported.

\end{description}

\ \\
\begin{minipage}{\linewidth}
  Example (importing of some attribute from the CVI dataset):\\
{
\begin{minted}[framesep=3mm, frame=single, tabsize=2, breaklines, breaksymbolleft=, fontsize=\footnotesize]{json}
{
    "attributes": [
        "M01.060",
        "C14.280",
        "C14.240"
    ],
    "datasource": "HospitalA"
}
\end{minted}
\captionof{lstlisting}{Example /smpc/import/categorical POST request body}
\label{sc:smpc-import-mesh}
}
\end{minipage}

\item[Response:] The server responds with \texttt{HTTP/1.1 200 OK} upon successful importing or with \texttt{HTTP/1.1 400 Bad Request} upon failure.
\end{description}

