\chapter{Sharemind: A Secure Computing Platform}\label{c:sharemind}

Sharemind \cite{bogdanov2008sharemind, bogdanov2013sharemind} is a general purpose SMPC system, for privacy preserving data processing operating on additively secret-shared values.
The idea of Sharemind is to provide a secure infrastructure that is able to host and evaluate privacy preserving algorithms.
Sharemind provides an easily programmable and flexible platform that enables non\hyp cryptographers to develop and test privacy preserving algorithms such as Privacy Preserving Data Mining (PPDM).
Sharemind is provably secure under the semi-honest (honest but curious) setting.

Sharemind’s is deployed as a distributed computation platform, that can be used both for data storage and computation.
The deployment model consists of (usually three) nodes, the computing nodes that use SMPC through secret sharing to privately process the data.
Secret sharing guarantees data confidentiality during storage.
All computations are done by the dedicated computing nodes.

Data providers submit their private inputs by sending the corresponding cryptographic shares to the computing nodes.
The secret sharing scheme ensures that each share is a random bit string.
Each host, computing node, is not able to decrypt the data, or even extract any information about them.
Consequently, a node holding that share learns no extra information about the data input (secret), than if they did not hold that share.
For that reason, data providers need not trust any of the computing nodes.
Instead providers must trust that the nodes as a group obey a set of rules such as that they do not collude during the computations.
In practice, the computing nodes will be servers run by independent entities, such as companies or government agencies, ideally having conflicting interests.
Data users want to analyze the information given by data providers.
Answers to their queries are given through the Sharemind system, and not from data providers directly.

The programmer of the privacy preserving applications does not necessarily know the underlying security protocols.
Application are developed using the SecreC programming language.
SecreC is an imperative domain specific language (DSL) that is syntactically similar to C.
The language uses a custom type system, which separates private / confidential data from public.
Apart from common data types, SecreC adds a security types which can be either public or private.
Public values are processed as usual, whereas private values, which are in secret-shared form are processed using secure computation.

