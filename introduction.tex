\chapter{Introduction}\label{c:introduction}
Big data is growing exponentially, 90 percent of which has been created in the past few years. People are constantly producing and publishing information about themselves.
Such data comes from browsing the web, talking to someone online, moving around emitting gps signal, being registered by cameras or credit card usage.
Many companies and organisations, such as Google and Facebook, make profit from big data analytics, collection and storage.

One of the most common examples is personalized advertisements, occurring from the massive data analytics.
It is becoming increasingly common for data about our location, music and movies we like, private conversations, and any other online trace we leave behind, to be linked to our purchasing preferences.

The aforementioned tracking and profiling comes from breaching of the individuals’ privacy.
Privacy is the ability of individuals to have control over how their personal information is collected and used, and thereby express themselves selectively.
Nowadays, the need of preserving someone's privacy is more crucial than ever. 

Financial information can be sensitive. Such information includes a person’s holdings, debts and transactions (\textit{e.g.} purchases).
This information, if compromised, can lead criminal activity such as fraud or identity theft.
Also, one’s purchases can be linked to places they visit, people they contact and so on; thus, such data should remain private. Another representative example that renders privacy of critical importance is medical data.
People may not be comfortable sharing their medical records to others, due to several reasons.
For instance, it could affect their employment, their insurance coverages, or people just do not want others to know about their medical or psychological conditions or treatments.
Medical data reveals a lot for a patient’s personal life and therefore should be protected.

An argument adopted by many, is that there is no need for privacy if you have nothing to hide.
This shows a failure to understand that privacy is a human right.
There is no need to justify why such a right is needed.
The burden of justification falls on the one seeking to violate this right.
Even when a right is not useful to you, you can't give away the right of others.
\textit{``Arguing that you don't care about the right to privacy because you have nothing to hide is no different than saying you don't care about free speech because you have nothing to say."} [Snowden].
