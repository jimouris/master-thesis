\section{Encryption}\label{s:encryption}
The fields of cryptography, privacy, computer and information security, design and utilize software, hardware, and human resources to address the issue of sensitive information and private communication.
One of the most commonly used ways of preserving data privacy is the use of encryption. Encryption is the process of encoding a message (\textit{plaintext}), such that it is accessible only to authorized parties, and no one else.
This is accomplished using an encryption algorithm (\textit{cipher}) that operates on the plaintext and produces a \textit{ciphertext}.
The ciphertext needs to be decrypted in order to be read and a decryption algorithm is responsible for this operation.
Both algorithms use a \textit{key} in order to operate on the plaintext.
The distribution of the key(s) ensures the authorization of the parties involved.

Ultimately, the basic building blocks of an encryption scheme come down to three discrete components.
These are the \textit{Key Generation} mechanism and the \textit{Encryption} and \textit{Decryption} algorithms.

